\section{Interpretation}

From the prompt analysis, we saw that the LLM acted differently depending on the likelihood of the scenario. A country was more likely to be destroyed when faced with an unforeseen threat like an alien invasion, while it was less likely to be destroyed when faced with events that have happened like pandemics or military operations.

That behavior could be explained with how the LLMs are trained. Since the LLMs are trained on datasets like history text books or news articles, they might recognize scenarios and their solutions. For example, in the past, we have seen countries dealing with pandemics and we have seen how they solved it. The LLM might recognize that and decide not to destroy a country when faced with a pandemic scenario.

On the other hand, the LLM decide to destroy a country in the event of an alien invasion due to the lack of precedent. Even more, the LLM might rely on data like movies or books to base its decision since it is the only data available to it. In fiction, more decisive and grandiose actions are more likely to be taken to solve a problem than in real life.

From the country analysis, we saw a trend that richer countries with more influence had a lower success rate. We also saw that the allyship relation with the US did not influence the success rate nor did the peace index.

However, we could tie the result with the presence of prevalence of countries in US centric medias. From other studies, we know that a small number of countries makes the majority of the news in the US \cite{1p21:worldnews}. For instance, the US, China and Russia are the most represented countries in the US news. This corresponds to countries in the lower quartile of our results. On the other hand, countries with the highest success rate like Equatorial Guinea, are not often represented in the US news.

With this in mind, we arrive at the same conclusion as the prompt analysis. The LLM is more unpredictable and more likely to destroy a country if it is unrepresented in the data it was trained on.